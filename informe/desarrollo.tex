\section{Desarrollo}
\paragraph{}
A continuaci\'on se presentan los pasos llevados a cabo para desarrollar el c\'odigo de cada uno de los ejercicios

\subsection{Ejercicio 1}
\begin{itemize}
 \item Habilitamos el Gate A20 y verificamos que est\'e habilitado para poder direccionar memoria m\'as all\'a del megabyte
 \item Armamos en un archivo en lenguaje C una Tabla Global de Descriptores (GDT) en la que definimos 4 descriptores de segmento: 
	\begin{itemize}
	\item segmento nulo
	\item segmento de c\'odigo
	\item segmento de datos
	\item segmento de memoria de video
	\end{itemize}

 \item Cargamos el registro de la GDT (GDTR) con la direccion base  y el l\'imite de la GDT que armamos en el paso previo.

 \item Copiamos el contenido Registro de Control 0 (CR0) y seteamos el primer bit (PE) para habilitar el modo protegido del procesador. Luego pisamos el contenido de CR0 con la copia modificada.

 \item Cargamos en el selector CS la posicion que tiene dentro de la GDT el descriptor del segmento de c\'odigo. Para ello realizamos, conforme explica el manual de Intel\footnote{}, un salto lejano a la posici\'on de una etiqueta donde comienza el c\'odigo de 32 bits a ejecutarse en modo protegido.

 \item Se le indica al compilador que la el c\'odigo que sigue es de 32 bits, y seguidamente actualizamos los dem\'as selectores (DS, FS, GS, SS) cargandoles la posicion que tiene dentro de la GDT el descriptor del segmento de datos.

 \item Cargamos en ES la posici\'on que tiene dentro de la GDT el descriptor del segmento de memoria de video para luego poder escribir por pantalla.
 
 \item Escribimos en pantalla un marco de '#'
\end{itemize}

\subsection{Ejercicio 2}
\begin{itemize}
 \item Definimos un Directorio de Tablas de Paginas para la tarea pintor con un único selector que apunta a la única Tabla de Paginas de la tarea. Del mismo modo hacemos lo propio con el Directorio de Tablas de Paginas de la tarea traductor y del kernel que comparten la misma Tabla de Paginas. En ambos directorios el resto de las entradas son definidas como nulas.
 \item Definimos una Tabla de Paginas de la tarea Pintor que realiza identity mapping en todas las páginas salvo en la pagina que empieza en la posicion 0x13000 (espacio de escritura del pintor) que es mapeada a pagina que inicia en la posicion 0xB8000 (memoria de video), y la pagina que empieza en la posición 0xB8000 que es mapeada a la pagina que inicia en la posicion 0x10000 (espacio de lectura del traductor). Por otra parte, se marcan todas las paginas como no presentes salvo las paginas correspondientes al kernel, a la GDT,IDT,TSS y rutinas de C, al espacio de escritura del pintor, a la pila del pintor y a la memoria de video del kernel
 \item 
 \item 
 \item 
 \item 
 \item 
\end{itemize}
